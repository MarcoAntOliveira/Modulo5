
    \documentclass[12pt,a4paper]{article}
    \usepackage{geometry}
    \geometry{paperwidth=210mm,paperheight=297mm,
    textwidth=150mm,textheight=210mm,
    top=30mm,bottom=30mm,
    left=30mm,right=30mm}

    \date{} % Remove a exibição da data
    \usepackage{xcolor}
    \usepackage{listings}
    \usepackage{graphicx}
    \usepackage{hyperref} % Para criar links
    \usepackage[utf8]{inputenc}
    \usepackage[T1]{fontenc}
    \usepackage[brazil]{babel}

    \lstset{
    language=python,
    basicstyle=\ttfamily,
    keywordstyle=\bfseries\color{blue},
    commentstyle=\color{blue},
    stringstyle=\color{red!70!black},
    numberstyle=\tiny,
    stepnumber=1,
    numbersep=5pt,
    backgroundcolor=\color{white},
    breaklines=true,
    breakautoindent=true,
    showspaces=false,
    showstringspaces=false,
    showtabs=false,
    tabsize=2,
    literate={~}{{\textasciitilde}}1, % Trata ~ como um caractere normal
    extendedchars=true, % Permite caracteres estendidos (acentos, etc.)
    inputencoding=utf8, % Define a codificação de entrada como UTF-8
    literate={á}{{\'a}}1 {ã}{{\~a}}1 {ç}{{\c{c}}}1
    }
    \title{Modulo 5}
    \begin{document}
    \maketitle
    \tableofcontents
    \newpage
    \section{Introdução ao PyQT5}
    \begin{lstlisting}
        # PySide6 para GUI (interface gráfica) com Qt em Python - Instalação
# - A seção original desse curso usou PyQt5 (estamos atualizando para PySide6)
# - Essas bibliotecas (PySide e PyQt) usam a biblioteca Qt
# - Qt e uma biblioteca usada para a criacao de GUI (interface grafica
#   do usuário) escrita em C++.
#   - PySide e PyQt conseguem fazer a ponte (binding) entre o Python e a
#   biblioteca para a criação de interfaces gráficas sem ter que usar outra
#   linguagem de programação.
# - PySide6 e uma referencia a versão 6 da Qt (Qt 6)
# - Qt e multiplataforma, ou seja, deve funcionar em Windows, Linux e Mac.

# Por que mudei de PyQt para PySide na atualização?
# - PySide foi desenvolvido pela The Qt Company (da Nokia), como parte do
#   projeto Qt for Python project - https://doc.qt.io/qtforpython/
# - Por usarem a mesma biblioteca (Qt), PySide e PyQt são extremamente
#   similares, muitas vezes os codigos são identicos. Portanto, mesmo que voce
#   ainda queira usar PyQt, sera muito simples portar os codigos. Muitas vezes
#   basta trocar o nome de PySide para PyQt e vice-versa.
# - A maior diferença entre PyQt e PySide está na licença:
#   PyQt usa GPL ou commercial e PySide usa LGPL.
#   Em resumo: com PySide, voce tem a permissao uso da biblioteca para fins
#   comerciais, sem ter que compartilhar o codigo escrito por voce para o
#   publico.
#   Licenças sao topicos complexos, portanto, se oriente sobre elas
#   antes de usar qualquer software de terceiros.
#   https://tldrlegal.com/license/gnu-lesser-general-public-license-v3-(lgpl-3)
    \end{lstlisting}

    \section{Instalando o PySide6 no seu ambiente virtual}
    \begin{lstlisting}
    #   Licenças sao topicos complexos, portanto, se oriente sobre elas
    #   antes de usar qualquer software de terceiros.
    #   https://tldrlegal.com/license/gnu-lesser-general-public-license-v3-(lgpl-3)
    import PySide6.QtCore

    # Prints PySide6 version
    print(PySide6.__version__)  # type: ignore

    # Prints the Qt version used to compile PySide6
    print(PySide6.QtCore.__version__)  # type: ignore
    \end{lstlisting}

    \section{QApplication e QPushButton de PySide6.QtWidgets}
    \begin{lstlisting}
        # QApplication e QPushButton de PySide6.QtWidgets
        # QApplication -> O Widget principal da aplicação
        # QPushButton -> Um botão
        # PySide6.QtWidgets -> Onde estão os widgets do PySide6
        import sys

        from PySide6.QtWidgets import QApplication, QPushButton

        app = QApplication(sys.argv)

        botao = QPushButton('Texto do botão')
        botao.setStyleSheet('font-size: 40px;')
        botao.show()  # Adiciona o widget na hierarquia e exibe a janela

        app.exec()  # O loop da aplicação
    \end{lstlisting}

    \section{QWidget e QLayout de PySide6.QtWidgets}
    \begin{lstlisting}
        # QWidget e QLayout de PySide6.QtWidgets
        # QWidget -> generico
        # QLayout -> Um widget de layout que recebe outros widgets
        import sys

        from PySide6.QtWidgets import QApplication, QGridLayout, QPushButton, QWidget

        app = QApplication(sys.argv)

        botao = QPushButton('Texto do botão')
        botao.setStyleSheet('font-size: 80px;')

        botao2 = QPushButton('Botão 2')
        botao2.setStyleSheet('font-size: 40px;')

        botao3 = QPushButton('Botão 3')
        botao3.setStyleSheet('font-size: 40px;')

        central_widget = QWidget()

        layout = QGridLayout()
        central_widget.setLayout(layout)

        layout.addWidget(botao, 1, 1, 1, 1)
        layout.addWidget(botao2, 1, 2, 1, 1)
        layout.addWidget(botao3, 3, 1, 1, 2)

        central_widget.show()  # Central widget entre na hierarquia e mostre sua janela
        app.exec()  # O loop da aplicação
            \end{lstlisting}

    \section{QMainWindow e centralWidget}
    \begin{lstlisting}
        # QMainWindow e centralWidget
        # -> QApplication (app)
        #   -> QMainWindow (window->setCentralWidget)
        #       -> CentralWidget (central_widget)
        #           -> Layout (layout)
        #               -> Widget 1 (botao1)
        #               -> Widget 2 (botao2)
        #               -> Widget 3 (botao3)
        #   -> show
        # -> exec
        import sys

        from PySide6.QtWidgets import (QApplication, QGridLayout, QMainWindow,
                                    QPushButton, QWidget)

        app = QApplication(sys.argv)
        window = QMainWindow()
        central_widget = QWidget()
        window.setCentralWidget(central_widget)
        window.setWindowTitle('Minha janela bonita')

        botao1 = QPushButton('Texto do botão')
        botao1.setStyleSheet('font-size: 80px;')

        botao2 = QPushButton('Botão 2')
        botao2.setStyleSheet('font-size: 40px;')

        botao3 = QPushButton('Botão 3')
        botao3.setStyleSheet('font-size: 40px;')

        layout = QGridLayout()
        central_widget.setLayout(layout)

        layout.addWidget(botao1, 1, 1, 1, 1)
        layout.addWidget(botao2, 1, 2, 1, 1)
        layout.addWidget(botao3, 3, 1, 1, 2)


        def slot_example(status_bar):
            status_bar.showMessage('O meu slot foi executado')


        # statusBar
        status_bar = window.statusBar()
        status_bar.showMessage('Mostrar mensagem na barra')

        # menuBar
        menu = window.menuBar()
        primeiro_menu = menu.addMenu('Primeiro menu')
        primeira_acao = primeiro_menu.addAction('Primeira ação')
        primeira_acao.triggered.connect(  # type:ignore
            lambda: slot_example(status_bar)
        )


        window.show()
        app.exec()  # O loop da aplicação
    \end{lstlisting}

    \section{O básico sobre Signal e Slots (eventos e documentação)}
    \begin{lstlisting}
        # O básico sobre Signal e Slots (eventos e documentação)
        import sys

        from PySide6.QtCore import Slot
        from PySide6.QtWidgets import (QApplication, QGridLayout, QMainWindow,
                                    QPushButton, QWidget)

        app = QApplication(sys.argv)
        window = QMainWindow()
        central_widget = QWidget()
        window.setCentralWidget(central_widget)
        window.setWindowTitle('Minha janela bonita')

        botao1 = QPushButton('Texto do botão')
        botao1.setStyleSheet('font-size: 80px;')

        botao2 = QPushButton('Botão 2')
        botao2.setStyleSheet('font-size: 40px;')

        botao3 = QPushButton('Botão 3')
        botao3.setStyleSheet('font-size: 40px;')

        layout = QGridLayout()
        central_widget.setLayout(layout)

        layout.addWidget(botao1, 1, 1, 1, 1)
        layout.addWidget(botao2, 1, 2, 1, 1)
        layout.addWidget(botao3, 3, 1, 1, 2)


        @Slot()
        def slot_example(status_bar):
            def inner():
                status_bar.showMessage('O meu slot foi executado')
            return inner


        @Slot()
        def outro_slot(checked):
            print('Está marcado?', checked)


        @Slot()
        def terceiro_slot(action):
            def inner():
                outro_slot(action.isChecked())
            return inner


        # statusBar
        status_bar = window.statusBar()
        status_bar.showMessage('Mostrar mensagem na barra')

        # menuBar
        menu = window.menuBar()
        primeiro_menu = menu.addMenu('Primeiro menu')
        primeira_acao = primeiro_menu.addAction('Primeira ação')
        primeira_acao.triggered.connect(slot_example(status_bar))  # type:ignore

        segunda_action = primeiro_menu.addAction('Segunda ação')
        segunda_action.setCheckable(True)
        segunda_action.toggled.connect(outro_slot)  # type:ignore
        segunda_action.hovered.connect(terceiro_slot(segunda_action))  # type:ignore

        botao1.clicked.connect(terceiro_slot(segunda_action))  # type:ignore

        window.show()
        app.exec()  # O loop da aplicação
    \end{lstlisting}

    \section{Trabalhando com classes e herança no PySide6}
    \begin{lstlisting}
        # Trabalhando com classes e herança no PySide6
        import sys
        
        from PySide6.QtCore import Slot
        from PySide6.QtWidgets import (QApplication, QGridLayout, QMainWindow,
                                       QPushButton, QWidget)
        
        
        class MyWindow(QMainWindow):
            def __init__(self, parent=None):
                super().__init__(parent)
        
                self.central_widget = QWidget()
        
                self.setCentralWidget(self.central_widget)
                self.setWindowTitle('Minha janela bonita')
        
                # Botão
                self.botao1 = self.make_button('Texto do botão')
                self.botao1.clicked.connect(self.segunda_acao_marcada)  # type: ignore
        
                self.botao2 = self.make_button('Botão 2')
        
                self.botao3 = self.make_button('Terceiro')
        
                self.grid_layout = QGridLayout()
                self.central_widget.setLayout(self.grid_layout)
        
                self.grid_layout.addWidget(self.botao1, 1, 1, 1, 1)
                self.grid_layout.addWidget(self.botao2, 1, 2, 1, 1)
                self.grid_layout.addWidget(self.botao3, 3, 1, 1, 2)
        
                # statusBar
                self.status_bar = self.statusBar()
                self.status_bar.showMessage('Mostrar mensagem na barra')
        
                # menuBar
                self.menu = self.menuBar()
                self.primeiro_menu = self.menu.addMenu('Primeiro menu')
                self.primeira_acao = self.primeiro_menu.addAction('Primeira ação')
                self.primeira_acao.triggered.connect(  # type:ignore
                    self.muda_mensagem_da_status_bar)
        
                self.segunda_action = self.primeiro_menu.addAction('Segunda ação')
                self.segunda_action.setCheckable(True)
                self.segunda_action.toggled.connect(  # type:ignore
                    self.segunda_acao_marcada)
                self.segunda_action.hovered.connect(  # type:ignore
                    self.segunda_acao_marcada)
        
            @Slot()
            def muda_mensagem_da_status_bar(self):
                self.status_bar.showMessage('O meu slot foi executado')
        
            @Slot()
            def segunda_acao_marcada(self):
                print('Está marcado?', self.segunda_action.isChecked())
        
            def make_button(self, text):
                btn = QPushButton(text)
                btn.setStyleSheet('font-size: 80px;')
                return btn
        
        
        if __name__ == '__main__':
            app = QApplication(sys.argv)
            window = MyWindow()
            window.show()
            app.exec()  # O loop da aplicação
    \end{lstlisting}
    \section{calculadora}
    \subsection{Criando a janela principal }
    \textbf{main.py}
    \begin{lstlisting}
        import sys

    from main_window import MainWindow
    from PySide6.QtWidgets import QApplication, QLabel

    if __name__ == '__main__':
        app = QApplication(sys.argv)
        window = MainWindow()

        label1 = QLabel('O meu texto')
        label1.setStyleSheet('font-size: 150px;')
        window.v_layout.addWidget(label1)
        window.adjustFixedSize()

        window.show()
        app.exec()
    \end{lstlisting}
    \textbf{mainWindow.py}
    \begin{lstlisting}
        from PySide6.QtWidgets import QMainWindow, QVBoxLayout, QWidget


    class MainWindow(QMainWindow):
    def __init__(self, parent: QWidget | None = None, *args, **kwargs) -> None:
        super().__init__(parent, *args, **kwargs)

        # Configurando o layout básico
        self.cw = QWidget()
        self.v_layout = QVBoxLayout()
        self.cw.setLayout(self.v_layout)
        self.setCentralWidget(self.cw)

        # Titulo da janela
        self.setWindowTitle('Calculadora')

    def adjustFixedSize(self):
        # ultima coisa a ser feita
        self.adjustSize()
        self.setFixedSize(self.width(), self.height())
    \end{lstlisting}


    \subsection{variáveis e método p/ adicionar widgets no vlayout}
    \begin{lstlisting}
        import sys

from main_window import MainWindow
from PySide6.QtWidgets import QApplication, QLabel
from PySide6.QtGui import QIcon
from PySide6.QtWidgets import QApplication
from variables import WINDOW_ICON_PATH

if __name__ == '__main__':
    # Cria a aplicação
    app = QApplication(sys.argv)
    window = MainWindow()

    label1 = QLabel('O meu texto')
    label1.setStyleSheet('font-size: 150px;')
    window.v_layout.addWidget(label1)
    window.adjustFixedSize()
    # Define o icone
    icon = QIcon(str(WINDOW_ICON_PATH))
    window.setWindowIcon(icon)
    app.setWindowIcon(icon)

    # Executa tudo
    window.adjustFixedSize()
    window.show()
    app.exec()
    \end{lstlisting}
    \textbf{MainWindow}
    \begin{lstlisting}
        # ultima coisa a ser feita
        self.adjustSize()
        self.setFixedSize(self.width(), self.height())

    def addWidgetToVLayout(self, widget: QWidget):
        self.v_layout.addWidget(widget)
    \end{lstlisting}
    \textbf{variables.py}
    \begin{lstlisting}
    from pathlib import Path

    ROOT_DIR = Path(__file__).parent
    FILES_DIR = ROOT_DIR / 'files'
    WINDOW_ICON_PATH = FILES_DIR / 'icon.png'
    \end{lstlisting}

    \subsection{Configurando o layout básico}
    \textbf{MainWindow.py}
    \begin{lstlisting}
        self.cw = QWidget()
        self.v_layout = QVBoxLayout()
        self.cw.setLayout(self.v_layout)
        self.vLayout = QVBoxLayout()
        self.cw.setLayout(self.vLayout)
        self.setCentralWidget(self.cw)

        # Titulo da janela
        def adjustFixedSize(self):
        self.setFixedSize(self.width(), self.height())

    def addWidgetToVLayout(self, widget: QWidget):
        self.v_layout.addWidget(widget)
        self.vLayout.addWidget(widget)
    \end{lstlisting}

    

    \subsection{QLineEdit e o display}
    \textbf{display.py}
    \begin{lstlisting}
        from PySide6.QtCore import Qt
        from PySide6.QtWidgets import QLineEdit
        from variables import BIG_FONT_SIZE, MINIMUM_WIDTH, TEXT_MARGIN


class Display(QLineEdit):
    def __init__(self, *args, **kwargs):
        super().__init__(*args, **kwargs)
        self.configStyle()

    def configStyle(self):
        margins = [TEXT_MARGIN for _ in range(4)]
        self.setStyleSheet(f'font-size: {BIG_FONT_SIZE}px;')
        self.setMinimumHeight(BIG_FONT_SIZE * 2)
        self.setMinimumWidth(MINIMUM_WIDTH)
        self.setAlignment(Qt.AlignmentFlag.AlignRight)
        self.setTextMargins(*margins)
    \end{lstlisting}

    \textbf{main.py}
    \begin{lstlisting}
        import sys

    from display import Display
    from main_window import MainWindow
    from PySide6.QtGui import QIcon
    from PySide6.QtWidgets import QApplication
    @@ -15,6 +16,10 @@
        window.setWindowIcon(icon)
        app.setWindowIcon(icon)
        
        # Display adicionado 
        display = Display()
        window.addToVLayout(display)

        # Executa tudo
        window.adjustFixedSize()
        window.show()
    \end{lstlisting}

    \textbf{mainWindow.py}
    \begin{lstlisting}
        self.adjustSize()
        self.setFixedSize(self.width(), self.height())

    def addWidgetToVLayout(self, widget: QWidget):
    def addToVLayout(self, widget: QWidget):
        self.vLayout.addWidget(widget)# new commit
    \end{lstlisting}
    \textbf{variables.py}
    \begin{lstlisting}
        ROOT_DIR = Path(__file__).parent
        FILES_DIR = ROOT_DIR / 'files'
        WINDOW_ICON_PATH = FILES_DIR / 'icon.png'

        # Sizing
        BIG_FONT_SIZE = 40
        MEDIUM_FONT_SIZE = 24
        SMALL_FONT_SIZE = 18
        TEXT_MARGIN = 15
        MINIMUM_WIDTH = 500
    \end{lstlisting}


    \subsection{criando um QLabel para mostrar informações
    main}
    \textbf{info.py}
    \begin{lstlisting}
        from PySide6.QtCore import Qt
from PySide6.QtWidgets import QLabel, QWidget
from variables import SMALL_FONT_SIZE


class Info(QLabel):
    def __init__(self, text: str, parent: QWidget | None = None) -> None:
        super().__init__(text, parent)
        self.configStyle()

    def configStyle(self):
        self.setStyleSheet(f'font-size: {SMALL_FONT_SIZE}px;')
        self.setAlignment(Qt.AlignmentFlag.AlignRight)
    \end{lstlisting}
    \textbf{main.py}
    \begin{lstlisting}
        import sys

from display import Display
from info import Info
from main_window import MainWindow
from PySide6.QtGui import QIcon
from PySide6.QtWidgets import QApplication
@@ -16,6 +17,10 @@
    window.setWindowIcon(icon)
    app.setWindowIcon(icon)

    # Info
    info = Info('2.0 ^ 10.0 = 1024')
    window.addToVLayout(info)

    # Display
    display = Display()
    window.addToVLayout(display)
    \end{lstlisting}

    \subsection{configurando o PyQt Dark Theme (qdarktheme) no PySide6}
    \textbf{main.py}
    \begin{lstlisting}
        from main_window import MainWindow
from PySide6.QtGui import QIcon
from PySide6.QtWidgets import QApplication
from styles import setupTheme
from variables import WINDOW_ICON_PATH

if __name__ == '__main__':
    # Cria a aplicação
    app = QApplication(sys.argv)
    setupTheme()
    window = MainWindow()

    # Define o icone
    \end{lstlisting}
    \textbf{styles.py}
    \begin{lstlisting}
        # QSS - Estilos do QT for Python
# https://doc.qt.io/qtforpython/tutorials/basictutorial/widgetstyling.html
# Dark Theme
# https://pyqtdarktheme.readthedocs.io/en/latest/how_to_use.html
import qdarktheme
from variables import (DARKER_PRIMARY_COLOR, DARKEST_PRIMARY_COLOR,
                       PRIMARY_COLOR)

qss = f"""
    PushButton[cssClass="specialButton"] {{
        color: #fff;
        background: {PRIMARY_COLOR};
    }}
    PushButton[cssClass="specialButton"]:hover {{
        color: #fff;
        background: {DARKER_PRIMARY_COLOR};
    }}
    PushButton[cssClass="specialButton"]:pressed {{
        color: #fff;
        background: {DARKEST_PRIMARY_COLOR};
    }}
"""


def setupTheme():
    qdarktheme.setup_theme(
        theme='dark',
        corner_shape='rounded',
        custom_colors={
            "[dark]": {
                "primary": f"{PRIMARY_COLOR}",
            },
            "[light]": {
                "primary": f"{PRIMARY_COLOR}",
            },
        },
        additional_qss=qss
    )
    \end{lstlisting}
    \textbf{variables.py}
    \begin{lstlisting}
        FILES_DIR = ROOT_DIR / 'files'
WINDOW_ICON_PATH = FILES_DIR / 'icon.png'

# Colors
(PRIMARY_COLOR = '#1e81b0'
DARKER_PRIMARY_COLOR = '#16658a'
DARKEST_PRIMARY_COLOR = '#115270'
)
# Sizing
BIG_FONT_SIZE = 40
MEDIUM_FONT_SIZE = 24
    \end{lstlisting}


    \subsection{ criando um botão com QPushButton no PySide6}
    \textbf{ criando um botão com QPushButton no PySide6}
    \begin{lstlisting}
        from PySide6.QtWidgets import QPushButton
from variables import MEDIUM_FONT_SIZE


class Button(QPushButton):
    def __init__(self, *args, **kwargs):
        super().__init__(*args, **kwargs)
        self.configStyle()

    def configStyle(self):
        font = self.font()
        font.setPixelSize(MEDIUM_FONT_SIZE)
        self.setFont(font)
        self.setMinimumSize(75, 75)
        self.setProperty('cssClass', 'specialButton')
    \end{lstlisting}

    \textbf{main.py}

    \begin{lstlisting}
        import sys

from buttons import Button
from display import Display
from info import Info
from main_window import MainWindow
@@ -27,6 +28,12 @@
    display = Display()
    window.addToVLayout(display)

    button = Button('Texto do botão')
    window.addToVLayout(button)

    button2 = Button('Texto do botão')
    window.addToVLayout(button2)

    # Executa tudo
    window.adjustFixedSize()
    window.show()
    \end{lstlisting}
    \textbf{styles.py}
    \begin{lstlisting}
        qss = f"""
    PushButton[cssClass="specialButton"] {{
    QPushButton[cssClass="specialButton"] {{
        color: #fff;
        background: {PRIMARY_COLOR};
    }}
    PushButton[cssClass="specialButton"]:hover {{
    QPushButton[cssClass="specialButton"]:hover {{
        color: #fff;
        background: {DARKER_PRIMARY_COLOR};
    }}
    PushButton[cssClass="specialButton"]:pressed {{
    QPushButton[cssClass="specialButton"]:pressed {{
        color: #fff;
        background: {DARKEST_PRIMARY_COLOR};
    }}
    \end{lstlisting}
    \subsection{grid de botões com QGridLayout}
    \textbf{buttons.py}
    \begin{lstlisting}
        from PySide6.QtWidgets import QPushButton
from PySide6.QtWidgets import QGridLayout, QPushButton
from variables import MEDIUM_FONT_SIZE


@@ -13,3 +13,16 @@ def configStyle(self):
        self.setFont(font)
        self.setMinimumSize(75, 75)
        self.setProperty('cssClass', 'specialButton')


class ButtonsGrid(QGridLayout):
    def __init__(self, *args, **kwargs) -> None:
        super().__init__(*args, **kwargs)

        self._grid_mask = [
            ['C', '', '^', '/'],
            ['7', '8', '9', '*'],
            ['4', '5', '6', '-'],
            ['1', '2', '3', '+'],
            ['',  '0', '.', '='],
        ]
    \end{lstlisting}

    \textbf{main.py}
    \begin{lstlisting}
        import sys

from buttons import Button
from buttons import Button, ButtonsGrid
from display import Display
from info import Info
from main_window import MainWindow
@@ -22,17 +22,15 @@

    # Info
    info = Info('2.0 ^ 10.0 = 1024')
    window.addToVLayout(info)
    window.addWidgetToVLayout(info)

    # Display
    display = Display()
    window.addToVLayout(display)
    window.addWidgetToVLayout(display)

    button = Button('Texto do botão')
    window.addToVLayout(button)

    button2 = Button('Texto do botão')
    window.addToVLayout(button2)
    # Grid
    buttonsGrid = ButtonsGrid()
    window.vLayout.addLayout(buttonsGrid)

    # Executa tudo
    window.adjustFixedSize()
    \end{lstlisting}
    \textbf{mainWindow.py}
    \begin{lstlisting}
        self.adjustSize()
        self.setFixedSize(self.width(), self.height())

    def addToVLayout(self, widget: QWidget):
    def addWidgetToVLayout(self, widget: QWidget):
        self.vLayout.addWidget(widget)
    \end{lstlisting}

    \subsection{ criando a grid de botões}
    \textbf{buttons.py}
    \begin{lstlisting}
        from PySide6.QtWidgets import QGridLayout, QPushButton
from utils import isEmpty, isNumOrDot
from variables import MEDIUM_FONT_SIZE


@@ -12,17 +13,27 @@ def configStyle(self):
        font.setPixelSize(MEDIUM_FONT_SIZE)
        self.setFont(font)
        self.setMinimumSize(75, 75)
        self.setProperty('cssClass', 'specialButton')


class ButtonsGrid(QGridLayout):
    def __init__(self, *args, **kwargs) -> None:
        super().__init__(*args, **kwargs)

        self._grid_mask = [
        self._gridMask = [
            ['C', '', '^', '/'],
            ['7', '8', '9', '*'],
            ['4', '5', '6', '-'],
            ['1', '2', '3', '+'],
            ['',  '0', '.', '='],
        ]
        self._makeGrid()

    def _makeGrid(self):
        for rowNumber, rowData in enumerate(self._gridMask):
            for colNumber, buttonText in enumerate(rowData):
                button = Button(buttonText)

                if not isNumOrDot(buttonText) and not isEmpty(buttonText):
                    button.setProperty('cssClass', 'specialButton')

                self.addWidget(button, rowNumber, colNumber)
    \end{lstlisting}
    \textbf{main.py}
    \begin{lstlisting}
        import sys

from buttons import Button, ButtonsGrid
from buttons import ButtonsGrid
from display import Display
from info import Info
from main_window import MainWindow
    \end{lstlisting}
    \textbf{utils.py}
    \begin{lstlisting}
        import re

NUM_OR_DOT_REGEX = re.compile(r'^[0-9.]$')


def isNumOrDot(string: str):
    return bool(NUM_OR_DOT_REGEX.search(string))


def isEmpty(string: str):
    return len(string) == 0
    \end{lstlisting}

    \subsection{criando um Slot com dados para o Signal clicked}
    \textbf{buttons.py}
    \begin{lstlisting}
        self.setCheckable(True)


        class ButtonsGrid(QGridLayout):
            def __init__(self, *args, **kwargs) -> None:
            def __init__(self, display: Display, *args, **kwargs) -> None:
                super().__init__(*args, **kwargs)
        
                self._gridMask = [
        @@ -26,6 +29,7 @@ def __init__(self, *args, **kwargs) -> None:
                    ['1', '2', '3', '+'],
                    ['',  '0', '.', '='],
                ]
                self.display = display
                self._makeGrid()
        
            def _makeGrid(self):
        @@ -37,3 +41,18 @@ def _makeGrid(self):
                            button.setProperty('cssClass', 'specialButton')
        
                        self.addWidget(button, rowNumber, colNumber)
                        buttonSlot = self._makeButtonDisplaySlot(
                            self._insertButtonTextToDisplay,
                            button,
                        )
                        button.clicked.connect(buttonSlot)  # type: ignore
        
            def _makeButtonDisplaySlot(self, func, *args, **kwargs):
                @Slot(bool)
                def realSlot(_):
                    func(*args, **kwargs)
                return realSlot
        
            def _insertButtonTextToDisplay(self, button):
                button_text = button.text()
                self.display.insert(button_text)
    \end{lstlisting}

    \textbf{main.py}
    \begin{lstlisting}
        window.addWidgetToVLayout(display)

    # Grid
    buttonsGrid = ButtonsGrid()
    buttonsGrid = ButtonsGrid(display)
    window.vLayout.addLayout(buttonsGrid)

    # Executa tudo
    \end{lstlisting}

    

    \subsection{permitindo apenas números válidos no display}
    \textbf{buttons.py}
    \begin{lstlisting}
        from display import Display
from PySide6.QtCore import Slot
from PySide6.QtWidgets import QGridLayout, QPushButton
from utils import isEmpty, isNumOrDot
from utils import isEmpty, isNumOrDot, isValidNumber
from variables import MEDIUM_FONT_SIZE


@@ -15,7 +15,6 @@ def configStyle(self):
        font.setPixelSize(MEDIUM_FONT_SIZE)
        self.setFont(font)
        self.setMinimumSize(75, 75)
        self.setCheckable(True)


class ButtonsGrid(QGridLayout):
@@ -54,5 +53,10 @@ def realSlot(_):
        return realSlot

    def _insertButtonTextToDisplay(self, button):
        button_text = button.text()
        self.display.insert(button_text)
        buttonText = button.text()
        newDisplayValue = self.display.text() + buttonText

        if not isValidNumber(newDisplayValue):
            return

        self.display.insert(buttonText)
    \end{lstlisting}

    \textbf{utils.py}
    \begin{lstlisting}
        return bool(NUM_OR_DOT_REGEX.search(string))


def isValidNumber(string: str):
    valid = False
    try:
        float(string)
        valid = True
    except ValueError:
        valid = False
    return valid


def isEmpty(string: str):
    return len(string) == 0
    \end{lstlisting}

    \subsection{Info (QLabel), TYPECHECKING, getter e setter }
    \textbf{buttons.py}
    \begin{lstlisting}
        from display import Display
from typing import TYPE_CHECKING

from PySide6.QtCore import Slot
from PySide6.QtWidgets import QGridLayout, QPushButton
from utils import isEmpty, isNumOrDot, isValidNumber
from variables import MEDIUM_FONT_SIZE

if TYPE_CHECKING:
    from display import Display
    from info import Info


class Button(QPushButton):
    def __init__(self, *args, **kwargs):
@@ -18,7 +23,9 @@ def configStyle(self):


class ButtonsGrid(QGridLayout):
    def __init__(self, display: Display, *args, **kwargs) -> None:
    def __init__(
            self, display: 'Display', info: 'Info', *args, **kwargs
    ) -> None:
        super().__init__(*args, **kwargs)

        self._gridMask = [
@@ -29,8 +36,19 @@ def __init__(self, display: Display, *args, **kwargs) -> None:
            ['',  '0', '.', '='],
        ]
        self.display = display
        self.info = info
        self._equation = ''
        self._makeGrid()

    @property
    def equation(self):
        return self._equation

    @equation.setter
    def equation(self, value):
        self._equation = value
        self.info.setText(value)

    def _makeGrid(self):
        for rowNumber, rowData in enumerate(self._gridMask):
            for colNumber, buttonText in enumerate(rowData):
    \end{lstlisting}

    \textbf{main.py}
    \begin{lstlisting}
        app.setWindowIcon(icon)

    # Info
    info = Info('2.0 ^ 10.0 = 1024')
    info = Info('Sua conta')
    window.addWidgetToVLayout(info)

    # Display
    display = Display()
    window.addWidgetToVLayout(display)

    # Grid
    buttonsGrid = ButtonsGrid(display)
    buttonsGrid = ButtonsGrid(display, info)
    window.vLayout.addLayout(buttonsGrid)

    # Executa tudo
    \end{lstlisting}

    \subsection{iniciando a configuração dos botões especiais}
    \textbf{buttons.py}
    \begin{lstlisting}
        if not isNumOrDot(buttonText) and not isEmpty(buttonText):
        button.setProperty('cssClass', 'specialButton')
        self._configSpecialButton(button)

    self.addWidget(button, rowNumber, colNumber)
    buttonSlot = self._makeButtonDisplaySlot(
        self._insertButtonTextToDisplay,
        button,
    )
    button.clicked.connect(buttonSlot)  # type: ignore
    slot = self._makeSlot(self._insertButtonTextToDisplay, button)
    self._connectButtonClicked(button, slot)

def _makeButtonDisplaySlot(self, func, *args, **kwargs):
def _connectButtonClicked(self, button, slot):
button.clicked.connect(slot)  # type: ignore

def _configSpecialButton(self, button):
text = button.text()

if text == 'C':
self._connectButtonClicked(button, self._clear)

def _makeSlot(self, func, *args, **kwargs):
@Slot(bool)
def realSlot(_):
func(*args, **kwargs)
@@ -78,3 +85,7 @@ def _insertButtonTextToDisplay(self, button):
return

self.display.insert(buttonText)

def _clear(self):
print('Vou fazer outra coisa aqui')
self.display.clear()
    \end{lstlisting}

\subsection{ botões especiais de operadores, clear e equation}
    \textbf{buttons.py}
    \begin{lstlisting}
        self.display = display
        self.info = info
        self._equation = ''
        self._equationInitialValue = 'Sua conta'
        self._left = None
        self._right = None
        self._op = None

        self.equation = self._equationInitialValue
        self._makeGrid()

    @property
@@ -71,8 +77,14 @@ def _configSpecialButton(self, button):
        if text == 'C':
            self._connectButtonClicked(button, self._clear)

        if text in '+-/*':
            self._connectButtonClicked(
                button,
                self._makeSlot(self._operatorClicked, button)
            )

    def _makeSlot(self, func, *args, **kwargs):
        @Slot(bool)
        @ Slot(bool)
        def realSlot(_):
            func(*args, **kwargs)
        return realSlot
@@ -87,5 +99,27 @@ def _insertButtonTextToDisplay(self, button):
        self.display.insert(buttonText)

    def _clear(self):
        print('Vou fazer outra coisa aqui')
        self._left = None
        self._right = None
        self._op = None
        self.equation = self._equationInitialValue
        self.display.clear()

    def _operatorClicked(self, button):
        buttonText = button.text()  # +-/* (etc...)
        displayText = self.display.text()  # Devera ser meu numero left
        self.display.clear()  # Limpa o display

        # Se a pessoa clicou no operador sem
        # configurar qualquer numero
        if not isValidNumber(displayText) and self._left is None:
            print('Não tem nada para colocar no valor da esquerda')
            return

        # Se houver algo no numero da esquerda,
        # nao fazemos nada. Aguardaremos o numero da direita.
        if self._left is None:
            self._left = float(displayText)

        self._op = buttonText
        self.equation = f'{self._left} {self._op} ??'
    \end{lstlisting}

\subsection{configurando o botão de igual e o número da direita}
    \textbf{buttons.py}
    \begin{lstlisting}
        self._makeSlot(self._operatorClicked, button)
        )

    if text in '=':
        self._connectButtonClicked(button, self._eq)

def _makeSlot(self, func, *args, **kwargs):
    @ Slot(bool)
    def realSlot(_):
@@ -123,3 +126,24 @@ def _operatorClicked(self, button):

    self._op = buttonText
    self.equation = f'{self._left} {self._op} ??'

def _eq(self):
    displayText = self.display.text()

    if not isValidNumber(displayText):
        print('Sem nada para a direita')
        return

    self._right = float(displayText)
    self.equation = f'{self._left} {self._op} {self._right}'
    result = 0.0 
        import math
from typing import TYPE_CHECKING

from PySide6.QtCore import Slot
@@ -77,7 +78,7 @@ def _configSpecialButton(self, button):
        if text == 'C':
            self._connectButtonClicked(button, self._clear)

        if text in '+-/*':
        if text in '+-/*^':
            self._connectButtonClicked(
                button,
                self._makeSlot(self._operatorClicked, button)
@@ -136,14 +137,22 @@ def _eq(self):

        self._right = float(displayText)
        self.equation = f'{self._left} {self._op} {self._right}'
        result = 0.0
        result = 'error'

        try:
            result = eval(self.equation)
            if '^' in self.equation and isinstance(self._left, float):
                result = math.pow(self._left, self._right)
            else:
                result = eval(self.equation)
        except ZeroDivisionError:
            print('Zero Division Error')
        except OverflowError:
            print('Numero muito grande')

        self.display.clear()
        self.info.setText(f'{self.equation} = {result}')
        self._left = result
        self._right = None

        if result == 'error':
            self._left = None
    \end{lstlisting}


\subsection{configurando o backspace do display no botão back}
    \textbf{buttons.py}
    \begin{lstlisting}
        super().__init__(*args, **kwargs)

        self._gridMask = [
            ['C', 'space', '^', '/'],
            ['C', 'D', '^', '/'],
            ['7', '8', '9', '*'],
            ['4', '5', '6', '-'],
            ['1', '2', '3', '+'],
@@ -78,6 +78,9 @@ def _configSpecialButton(self, button):
        if text == 'C':
            self._connectButtonClicked(button, self._clear)

        if text in 'D':
            self._connectButtonClicked(button, self.display.backspace)

        if text in '+-/*^':
            self._connectButtonClicked(
                button,
    \end{lstlisting}

    
\subsection{diálogos com o usuário com QMessageBox}
    \textbf{buttons.py}
    \begin{lstlisting}
        if TYPE_CHECKING:
    from display import Display
    from info import Info
    from main_window import MainWindow


class Button(QPushButton):
@@ -25,7 +26,8 @@ def configStyle(self):

class ButtonsGrid(QGridLayout):
    def __init__(
            self, display: 'Display', info: 'Info', *args, **kwargs
            self, display: 'Display', info: 'Info', window: 'MainWindow',
            *args, **kwargs
    ) -> None:
        super().__init__(*args, **kwargs)

@@ -38,6 +40,7 @@ def __init__(
        ]
        self.display = display
        self.info = info
        self.window = window
        self._equation = ''
        self._equationInitialValue = 'Sua conta'
        self._left = None
@@ -120,7 +123,7 @@ def _operatorClicked(self, button):
        # Se a pessoa clicou no operador sem
        # configurar qualquer numero
        if not isValidNumber(displayText) and self._left is None:
            print('Não tem nada para colocar no valor da esquerda')
            self._showError('Voce nao digitou nada.')
            return

        # Se houver algo no numero da esquerda,
@@ -135,7 +138,7 @@ def _eq(self):
        displayText = self.display.text()

        if not isValidNumber(displayText):
            print('Sem nada para a direita')
            self._showError('Conta incompleta.')
            return

        self._right = float(displayText)
@@ -148,9 +151,9 @@ def _eq(self):
            else:
                result = eval(self.equation)
        except ZeroDivisionError:
            print('Zero Division Error')
            self._showError('Divisão por zero.')
        except OverflowError:
            print('Numero muito grande')
            self._showError('Essa conta não pode ser realizada.')

        self.display.clear()
        self.info.setText(f'{self.equation} = {result}')
@@ -159,3 +162,18 @@ def _eq(self):

        if result == 'error':
            self._left = None

    def _makeDialog(self, text):
        msgBox = self.window.makeMsgBox()
        msgBox.setText(text)
        return msgBox

    def _showError(self, text):
        msgBox = self._makeDialog(text)
        msgBox.setIcon(msgBox.Icon.Critical)
        msgBox.exec()

    def _showInfo(self, text):
        msgBox = self._makeDialog(text)
        msgBox.setIcon(msgBox.Icon.Information)
        msgBox.exec()
    \end{lstlisting}

    \textbf{main.py}
    \begin{lstlisting}
        window.addWidgetToVLayout(display)

        # Grid
        buttonsGrid = ButtonsGrid(display, info)
        buttonsGrid = ButtonsGrid(display, info, window)
        window.vLayout.addLayout(buttonsGrid)

        # Executa tudo
    \end{lstlisting}

    \textbf{mainWindow.py}
    \begin{lstlisting}
        from PySide6.QtWidgets import QMainWindow, QVBoxLayout, QWidget
        from PySide6.QtWidgets import QMainWindow, QMessageBox, QVBoxLayout, QWidget


        class MainWindow(QMainWindow):
        @@ -21,3 +21,6 @@ def adjustFixedSize(self):

        def addWidgetToVLayout(self, widget: QWidget):
            self.vLayout.addWidget(widget)

        def makeMsgBox(self):
            return QMessageBox(self)
    \end{lstlisting}
    \section{Criando e compilando um arquivo UI com o Qt Designer}
        \textbf{aula203-qtdesigner/src/window.py}
        \begin{lstlisting}
            # -*- coding: utf-8 -*-


## Form generated from reading UI file 'window.ui'
##
## Created by: Qt User Interface Compiler version 6.4.2
##
## WARNING! All changes made in this file will be lost when recompiling UI file!


from PySide6.QtCore import (QCoreApplication, QDate, QDateTime, QLocale,
    QMetaObject, QObject, QPoint, QRect,
    QSize, QTime, QUrl, Qt)
from PySide6.QtGui import (QBrush, QColor, QConicalGradient, QCursor,
    QFont, QFontDatabase, QGradient, QIcon,
    QImage, QKeySequence, QLinearGradient, QPainter,
    QPalette, QPixmap, QRadialGradient, QTransform)
from PySide6.QtWidgets import (QApplication, QGridLayout, QHBoxLayout, QLabel,
    QLineEdit, QMainWindow, QMenuBar, QPushButton,
    QSizePolicy, QStatusBar, QWidget)

class Ui_MainWindow(object):
    def setupUi(self, MainWindow):
        if not MainWindow.objectName():
            MainWindow.setObjectName(u"MainWindow")
        MainWindow.resize(800, 600)
        self.centralwidget = QWidget(MainWindow)
        self.centralwidget.setObjectName(u"centralwidget")
        self.horizontalLayout = QHBoxLayout(self.centralwidget)
        self.horizontalLayout.setObjectName(u"horizontalLayout")
        self.gridLayout = QGridLayout()
        self.gridLayout.setObjectName(u"gridLayout")
        self.labelResult = QLabel(self.centralwidget)
        self.labelResult.setObjectName(u"labelResult")
        font = QFont()
        font.setPointSize(40)
        self.labelResult.setFont(font)
        self.labelResult.setAlignment(Qt.AlignCenter)

        self.gridLayout.addWidget(self.labelResult, 0, 0, 1, 1)

        self.gridLayout_2 = QGridLayout()
        self.gridLayout_2.setObjectName(u"gridLayout_2")
        self.labelName = QLabel(self.centralwidget)
        self.labelName.setObjectName(u"labelName")
        font1 = QFont()
        font1.setPointSize(30)
        self.labelName.setFont(font1)

        self.gridLayout_2.addWidget(self.labelName, 0, 0, 1, 1)

        self.lineName = QLineEdit(self.centralwidget)
        self.lineName.setObjectName(u"lineName")
        self.lineName.setFont(font1)

        self.gridLayout_2.addWidget(self.lineName, 0, 1, 1, 1)

        self.buttonSend = QPushButton(self.centralwidget)
        self.buttonSend.setObjectName(u"buttonSend")
        self.buttonSend.setFont(font1)

        self.gridLayout_2.addWidget(self.buttonSend, 0, 2, 1, 1)


        self.gridLayout.addLayout(self.gridLayout_2, 1, 0, 1, 1)


        self.horizontalLayout.addLayout(self.gridLayout)

        MainWindow.setCentralWidget(self.centralwidget)
        self.menubar = QMenuBar(MainWindow)
        self.menubar.setObjectName(u"menubar")
        self.menubar.setGeometry(QRect(0, 0, 800, 22))
        MainWindow.setMenuBar(self.menubar)
        self.statusbar = QStatusBar(MainWindow)
        self.statusbar.setObjectName(u"statusbar")
        MainWindow.setStatusBar(self.statusbar)

        self.retranslateUi(MainWindow)

        QMetaObject.connectSlotsByName(MainWindow)
    # setupUi

    def retranslateUi(self, MainWindow):
        MainWindow.setWindowTitle(QCoreApplication.translate("MainWindow", u"MainWindow", None))
        self.labelResult.setText(QCoreApplication.translate("MainWindow", u"Voltei!", None))
        self.labelName.setText(QCoreApplication.translate("MainWindow", u"Seu nome:", None))
        self.lineName.setPlaceholderText(QCoreApplication.translate("MainWindow", u"Digite seu nome", None))
        self.buttonSend.setText(QCoreApplication.translate("MainWindow", u"Enviar", None))
    # retranslateUi
        \end{lstlisting}
        \textbf{aula203qtdesigner/ui/uiwindow.py}
        \begin{lstlisting}
            # -*- coding: utf-8 -*-


from PySide6.QtCore import (QCoreApplication, QDate, QDateTime, QLocale,
    QMetaObject, QObject, QPoint, QRect,
    QSize, QTime, QUrl, Qt)
from PySide6.QtGui import (QBrush, QColor, QConicalGradient, QCursor,
    QFont, QFontDatabase, QGradient, QIcon,
    QImage, QKeySequence, QLinearGradient, QPainter,
    QPalette, QPixmap, QRadialGradient, QTransform)
from PySide6.QtWidgets import (QApplication, QGridLayout, QHBoxLayout, QLabel,
    QLineEdit, QMainWindow, QMenuBar, QPushButton,
    QSizePolicy, QStatusBar, QWidget)

class UiMainWindow(object):
    def setupUi(self, MainWindow):
        if not MainWindow.objectName():
            MainWindow.setObjectName(u"MainWindow")
        MainWindow.resize(800, 600)
        self.centralwidget = QWidget(MainWindow)
        self.centralwidget.setObjectName(u"centralwidget")
        self.horizontalLayout = QHBoxLayout(self.centralwidget)
        self.horizontalLayout.setObjectName(u"horizontalLayout")
        self.gridLayout = QGridLayout()
        self.gridLayout.setObjectName(u"gridLayout")
        self.labelResult = QLabel(self.centralwidget)
        self.labelResult.setObjectName(u"labelResult")
        font = QFont()
        font.setPointSize(40)
        self.labelResult.setFont(font)
        self.labelResult.setAlignment(Qt.AlignCenter)

        self.gridLayout.addWidget(self.labelResult, 0, 0, 1, 1)

        self.gridLayout2 = QGridLayout()
        self.gridLayout2.setObjectName(u"gridLayout2")
        self.labelName = QLabel(self.centralwidget)
        self.labelName.setObjectName(u"labelName")
        font1 = QFont()
        font1.setPointSize(30)
        self.labelName.setFont(font1)

        self.gridLayout2.addWidget(self.labelName, 0, 0, 1, 1)

        self.lineName = QLineEdit(self.centralwidget)
        self.lineName.setObjectName(u"lineName")
        self.lineName.setFont(font1)

        self.gridLayout2.addWidget(self.lineName, 0, 1, 1, 1)

        self.buttonSend = QPushButton(self.centralwidget)
        self.buttonSend.setObjectName(u"buttonSend")
        self.buttonSend.setFont(font1)

        self.gridLayout2.addWidget(self.buttonSend, 0, 2, 1, 1)


        self.gridLayout.addLayout(self.gridLayout2, 1, 0, 1, 1)


        self.horizontalLayout.addLayout(self.gridLayout)

        MainWindow.setCentralWidget(self.centralwidget)
        self.menubar = QMenuBar(MainWindow)
        self.menubar.setObjectName(u"menubar")
        self.menubar.setGeometry(QRect(0, 0, 800, 22))
        MainWindow.setMenuBar(self.menubar)
        self.statusbar = QStatusBar(MainWindow)
        self.statusbar.setObjectName(u"statusbar")
        MainWindow.setStatusBar(self.statusbar)

        self.retranslateUi(MainWindow)

        QMetaObject.connectSlotsByName(MainWindow)
     setupUi

    def retranslateUi(self, MainWindow):
        MainWindow.setWindowTitle(QCoreApplication.translate("MainWindow", u"MainWindow", None))
        self.labelResult.setText(QCoreApplication.translate("MainWindow", u"Voltei!", None))
        self.labelName.setText(QCoreApplication.translate("MainWindow", u"Seu nome:", None))
        self.lineName.setPlaceholderText(QCoreApplication.translate("MainWindow", u"Digite seu nome", None))
        self.buttonSend.setText(QCoreApplication.translate("MainWindow", u"Enviar", None))
     retranslateUi
        \end{lstlisting}
        \textbf{aula203-qtdesigner/ui/window.ui}
        \begin{lstlisting}
            <?xml version="1.0" encoding="UTF-8"?>
<ui version="4.0">
 <class>MainWindow</class>
 <widget class="QMainWindow" name="MainWindow">
  <property name="geometry">
   <rect>
    <x>0</x>
    <y>0</y>
    <width>800</width>
    <height>600</height>
   </rect>
  </property>
  <property name="windowTitle">
   <string>MainWindow</string>
  </property>
  <widget class="QWidget" name="centralwidget">
   <layout class="QHBoxLayout" name="horizontalLayout">
    <item>
     <layout class="QGridLayout" name="gridLayout">
      <item row="0" column="0">
       <widget class="QLabel" name="labelResult">
        <property name="font">
         <font>
          <pointsize>40</pointsize>
         </font>
        </property>
        <property name="text">
         <string>Voltei!</string>
        </property>
        <property name="alignment">
         <set>Qt::AlignCenter</set>
        </property>
       </widget>
      </item>
      <item row="1" column="0">
       <layout class="QGridLayout" name="gridLayout_2">
        <item row="0" column="0">
         <widget class="QLabel" name="labelName">
          <property name="font">
           <font>
            <pointsize>30</pointsize>
           </font>
          </property>
          <property name="text">
           <string>Seu nome:</string>
          </property>
         </widget>
        </item>
        <item row="0" column="1">
         <widget class="QLineEdit" name="lineName">
          <property name="font">
           <font>
            <pointsize>30</pointsize>
           </font>
          </property>
          <property name="placeholderText">
           <string>Digite seu nome</string>
          </property>
         </widget>
        </item>
        <item row="0" column="2">
         <widget class="QPushButton" name="buttonSend">
          <property name="font">
           <font>
            <pointsize>30</pointsize>
           </font>
          </property>
          <property name="text">
           <string>Enviar</string>
          </property>
         </widget>
        </item>
       </layout>
      </item>
     </layout>
    </item>
   </layout>
  </widget>
  <widget class="QMenuBar" name="menubar">
   <property name="geometry">
    <rect>
     <x>0</x>
     <y>0</y>
     <width>800</width>
     <height>22</height>
    </rect>
   </property>
  </widget>
  <widget class="QStatusBar" name="statusbar"/>
 </widget>
 <resources/>
 <connections/>
</ui>
        \end{lstlisting}
    \section{Usando um arquivo UI do Qt Designer com seu código Python}
    \textbf{aula203qtdesigner/src/mainwindow.py}
    \begin{lstlisting}
        import sys

from PySide6.QtWidgets import QApplication, QMainWindow
from window import Ui_MainWindow


class MainWindow(QMainWindow, Ui_MainWindow):
    def __init__(self, parent=None):
        super().__init__(parent)
        self.setupUi(self)

        self.buttonSend.clicked.connect(self.changeLabelResult)  # type: ignore

    def changeLabelResult(self):
        text = self.lineName.text()
        self.labelResult.setText(text)


if __name__ == '__main__':
    app = QApplication(sys.argv)
    mainWindow = MainWindow()
    mainWindow.show()
    app.exec()
    \end{lstlisting}
    \section{QObject e QThread}
    \subsection{criando a janela inicial com Qt Designer}
    \textbf{main.py}
    \begin{lstlisting}
        import sys

from PySide6.QtWidgets import QApplication, QWidget
from ui_workerui import Ui_myWidget


class MyWidget(QWidget, Ui_myWidget):
    def __init__(self, parent=None):
        super().__init__(parent)
        self.setupUi(self)


if __name__ == '__main__':
    app = QApplication(sys.argv)
    myWidget = MyWidget()
    myWidget.show()
    app.exec()
    \end{lstlisting}

    \textbf{uiworkerui.py}
    \begin{lstlisting}
        from PySide6.QtCore import (QCoreApplication, QDate, QDateTime, QLocale,
    QMetaObject, QObject, QPoint, QRect,
    QSize, QTime, QUrl, Qt)
from PySide6.QtGui import (QBrush, QColor, QConicalGradient, QCursor,
    QFont, QFontDatabase, QGradient, QIcon,
    QImage, QKeySequence, QLinearGradient, QPainter,
    QPalette, QPixmap, QRadialGradient, QTransform)
from PySide6.QtWidgets import (QApplication, QGridLayout, QHBoxLayout, QLabel,
    QPushButton, QSizePolicy, QWidget)

class Ui_myWidget(object):
    def setupUi(self, myWidget):
        if not myWidget.objectName():
            myWidget.setObjectName(u"myWidget")
        myWidget.resize(400, 300)
        font = QFont()
        font.setPointSize(40)
        myWidget.setFont(font)
        self.horizontalLayout = QHBoxLayout(myWidget)
        self.horizontalLayout.setObjectName(u"horizontalLayout")
        self.gridLayout = QGridLayout()
        self.gridLayout.setObjectName(u"gridLayout")
        self.label2 = QLabel(myWidget)
        self.label2.setObjectName(u"label2")

        self.gridLayout.addWidget(self.label2, 0, 1, 1, 1)

        self.label1 = QLabel(myWidget)
        self.label1.setObjectName(u"label1")

        self.gridLayout.addWidget(self.label1, 0, 0, 1, 1)

        self.button1 = QPushButton(myWidget)
        self.button1.setObjectName(u"button1")

        self.gridLayout.addWidget(self.button1, 1, 0, 1, 1)

        self.button2 = QPushButton(myWidget)
        self.button2.setObjectName(u"button2")

        self.gridLayout.addWidget(self.button2, 1, 1, 1, 1)


        self.horizontalLayout.addLayout(self.gridLayout)


        self.retranslateUi(myWidget)

        QMetaObject.connectSlotsByName(myWidget)
    # setupUi

    def retranslateUi(self, myWidget):
        myWidget.setWindowTitle(QCoreApplication.translate("myWidget", u"Form", None))
        self.label2.setText(QCoreApplication.translate("myWidget", u"L2", None))
        self.label1.setText(QCoreApplication.translate("myWidget", u"L1", None))
        self.button1.setText(QCoreApplication.translate("myWidget", u"B1", None))
        self.button2.setText(QCoreApplication.translate("myWidget", u"B2", None))
    # retranslateUi

    \end{lstlisting}
    \textbf{workerui.ui}
    \begin{lstlisting}
        <?xml version="1.0" encoding="UTF-8"?>
<ui version="4.0">
 <class>myWidget</class>
 <widget class="QWidget" name="myWidget">
  <property name="geometry">
   <rect>
    <x>0</x>
    <y>0</y>
    <width>400</width>
    <height>300</height>
   </rect>
  </property>
  <property name="font">
   <font>
    <pointsize>40</pointsize>
   </font>
  </property>
  <property name="windowTitle">
   <string>Form</string>
  </property>
  <layout class="QHBoxLayout" name="horizontalLayout">
   <item>
    <layout class="QGridLayout" name="gridLayout">
     <item row="0" column="1">
      <widget class="QLabel" name="label2">
       <property name="text">
        <string>L2</string>
       </property>
      </widget>
     </item>
     <item row="0" column="0">
      <widget class="QLabel" name="label1">
       <property name="text">
        <string>L1</string>
       </property>
      </widget>
     </item>
     <item row="1" column="0">
      <widget class="QPushButton" name="button1">
       <property name="text">
        <string>B1</string>
       </property>
      </widget>
     </item>
     <item row="1" column="1">
      <widget class="QPushButton" name="button2">
       <property name="text">
        <string>B2</string>
       </property>
      </widget>
     </item>
    </layout>
   </item>
  </layout>
 </widget>
 <resources/>
 <connections/>
</ui>
    \end{lstlisting}



    \subsection{ criando o Worker}
    \textbf{main.py}
    \begin{lstlisting}
        import sys
import time

from PySide6.QtCore import QObject, Signal, Slot
from PySide6.QtWidgets import QApplication, QWidget
from uiworkerui import UimyWidget


class Worker1(QObject):
    started = Signal(str)
    progressed = Signal(str)
    finished = Signal(str)

    def run(self):
        value = '0'
        self.started.emit(value)
        for i in range(5):
            value = str(i)
            self.progressed.emit(value)
            time.sleep(1)
        self.finished.emit(value)


class MyWidget(QWidget, Ui_myWidget):
    def __init__(self, parent=None):
        super().__init__(parent)
        self.setupUi(self)

        self.button1.clicked.connect(self.hardWork1)
        self.button2.clicked.connect(self.hardWork2)

    def hardWork1(self):
        self.label1.setText('1 terminado')

    def hardWork2(self):
        for i in range(5):
            print(i)
            time.sleep(1)
        self.label2.setText('2 terminado')


if __name__ == '__main__':
    app = QApplication(sys.argv)
    \end{lstlisting}

    \subsection{movendo workers para threads separadas}
    \textbf{main.py}
    \begin{lstlisting}
        import sys
import time

from PySide6.QtCore import QObject, Signal, Slot
from PySide6.QtCore import QObject, QThread, Signal
from PySide6.QtWidgets import QApplication, QWidget
from ui_workerui import Ui_myWidget

@@ -11,7 +11,7 @@ class Worker1(QObject):
    progressed = Signal(str)
    finished = Signal(str)

    def run(self):
    def doWork(self):
        value = '0'
        self.started.emit(value)
        for i in range(5):
@@ -30,13 +30,78 @@ def __init__(self, parent=None):
        self.button2.clicked.connect(self.hardWork2)

    def hardWork1(self):
        self.label1.setText('1 terminado')
        self._worker = Worker1()
        self._thread = QThread()

        worker = self._worker
        thread = self._thread

        # Mover o worker para a thread
        worker.moveToThread(thread)

        # Run
        thread.started.connect(worker.doWork)
        worker.finished.connect(thread.quit)

        thread.finished.connect(thread.deleteLater)
        worker.finished.connect(worker.deleteLater)

        worker.started.connect(self.worker1Started)
        worker.progressed.connect(self.worker1Progressed)
        worker.finished.connect(self.worker1Finished)

        thread.start()

    def worker1Started(self, value):
        self.button1.setDisabled(True)
        self.label1.setText(value)
        print('worker iniciado')

    def worker1Progressed(self, value):
        self.label1.setText(value)
        print('em progresso')

    def worker1Finished(self, value):
        self.label1.setText(value)
        self.button1.setDisabled(False)
        print('worker finalizado')

    def hardWork2(self):
        for i in range(5):
            print(i)
            time.sleep(1)
        self.label2.setText('2 terminado')
        self._worker2 = Worker1()
        self._thread2 = QThread()

        worker = self._worker2
        thread = self._thread2

        # Mover o worker para a thread
        worker.moveToThread(thread)

        # Run
        thread.started.connect(worker.doWork)
        worker.finished.connect(thread.quit)

        thread.finished.connect(thread.deleteLater)
        worker.finished.connect(worker.deleteLater)

        worker.started.connect(self.worker2Started)
        worker.progressed.connect(self.worker2Progressed)
        worker.finished.connect(self.worker2Finished)

        thread.start()

    def worker2Started(self, value):
        self.button2.setDisabled(True)
        self.label2.setText(value)
        print('worker 2 iniciado')

    def worker2Progressed(self, value):
        self.label2.setText(value)
        print('2 em progresso')

    def worker2Finished(self, value):
        self.label2.setText(value)
        self.button2.setDisabled(False)
        print('2 worker finalizado')


if __name__ == '__main__':
    \end{lstlisting}

    \subsection{ código comentado}
    \textbf{main.py}
    \begin{lstlisting}
        self.finished.emit(value)


        class Worker2(QObject):
            started = Signal(str)
            progressed = Signal(str)
            finished = Signal(str)
        
            def executeMe(self):
                value = '0'
                self.started.emit(value)
                for i in range(50, 100, 5):
                    value = str(i)
                    self.progressed.emit(value)
                    time.sleep(0.3)
                self.finished.emit(value)
        
        
        class MyWidget(QWidget, Ui_myWidget):
            def __init__(self, parent=None):
                super().__init__(parent)
                self.setupUi(self)
        
                self.button1.clicked.connect(self.hardWork1)
                self.button2.clicked.connect(self.hardWork2)
                self.button1.clicked.connect(self.hardWork1)  # type: ignore
                self.button2.clicked.connect(self.hardWork2)  # type: ignore
        
            def hardWork1(self):
                self._worker = Worker1()
                self._thread = QThread()
                self._worker1 = Worker1()
                self._thread1 = QThread()
        
                worker = self._worker
                thread = self._thread
                 Isso garante que o widget vai ter uma referencia para worker e thread
                worker = self._worker1
                thread = self._thread1
        
                # Mover o worker para a thread
                # Worker e movido para a thread. Todas as funçoes e metodos do
                # objeto de worker serão executados na thread criado pela QThread.
                worker.moveToThread(thread)
        
                # Run
                thread.started.connect(worker.doWork)
                # Quando uma QThread e iniciada, emite o sinal started automaticamente.
                thread.started.connect(worker.doWork)  # type: ignore
        
                # O sinal finished e emitido pelo objeto worker quando o trabalho que
                # ele esta executando e concluido. Isso aciona o metodo quit da qthread
                # que interrompe o loop de eventos dela.
                worker.finished.connect(thread.quit)
        
                thread.finished.connect(thread.deleteLater)
                # deleteLater solicita a exclusão do objeto worker do sistema de
                # gerenciamento de memoria do Python. Quando o worker finaliza, ele
                # emite um sinal finished que vai executar o metodo deleteLater.
                # Isso garante que objetos sejam removidos da memoria corretamente.
                thread.finished.connect(thread.deleteLater)  # type: ignore
                worker.finished.connect(worker.deleteLater)
        
                # Aqui estao seus metodos e inicio, meio e fim
                # execute o que quiser
                worker.started.connect(self.worker1Started)
                worker.progressed.connect(self.worker1Progressed)
                worker.finished.connect(self.worker1Finished)
        
                # Inicie a thread
                thread.start()
        
            def worker1Started(self, value):
                self.button1.setDisabled(True)
                self.label1.setText(value)
                print('worker iniciado')
                print('worker 1 iniciado', value)
        
            def worker1Progressed(self, value):
                self.label1.setText(value)
                print('em progresso')
                print('1 em progresso', value)
        
            def worker1Finished(self, value):
                self.label1.setText(value)
                self.button1.setDisabled(False)
                print('worker finalizado')
                print('worker 1 finalizado', value)
        
            def hardWork2(self):
                self._worker2 = Worker1()
                self._worker2 = Worker2()
                self._thread2 = QThread()
        
                # Isso garante que o widget vai ter uma referencia para worker e thread
                worker = self._worker2
                thread = self._thread2
        
                # Mover o worker para a thread
                # Worker e movido para a thread. Todas as funçoes e metodos do
                # objeto de worker serão executados na thread criado pela QThread.
                worker.moveToThread(thread)
        
                # Run
                thread.started.connect(worker.doWork)
                # Quando uma QThread e iniciada, emite o sinal started automaticamente.
                # Nome do metodo "doWork" modificado para "executeMe" (p/ exemplo)
                thread.started.connect(worker.executeMe)  # type: ignore
        
                # O sinal finished e emitido pelo objeto worker quando o trabalho que
                # ele esta executando e concluido. Isso aciona o metodo quit da qthread
                # que interrompe o loop de eventos dela.
                worker.finished.connect(thread.quit)
        
                thread.finished.connect(thread.deleteLater)
                # deleteLater solicita a exclusão do objeto worker do sistema de
                # gerenciamento de memoria do Python. Quando o worker finaliza, ele
                # emite um sinal finished que vai executar o metodo deleteLater.
                # Isso garante que objetos sejam removidos da memoria corretamente.
                thread.finished.connect(thread.deleteLater)  # type: ignore
                worker.finished.connect(worker.deleteLater)
        
                # Aqui estão seus metodos e inicio, meio e fim
                # execute o que quiser
                worker.started.connect(self.worker2Started)
                worker.progressed.connect(self.worker2Progressed)
                worker.finished.connect(self.worker2Finished)
        
                # Inicie a thread
                thread.start()
        
            def worker2Started(self, value):
                self.button2.setDisabled(True)
                self.label2.setText(value)
                print('worker 2 iniciado')
                print('worker 2 iniciado', value)
        
            def worker2Progressed(self, value):
                self.label2.setText(value)
                print('2 em progresso')
                print('2 em progresso', value)
        
            def worker2Finished(self, value):
                self.label2.setText(value)
                self.button2.setDisabled(False)
                print('2 worker finalizado')
                print('worker 2 finalizado', value)
        
        
        if __name__ == '__main__':
    \end{lstlisting}







    \end{document}
    
    \section{}
    \textbf{}
    \begin{lstlisting}
    \end{lstlisting}

